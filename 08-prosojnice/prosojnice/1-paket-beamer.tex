%  Naslov prosojnice lahko naredimo tudi z dodatnim parametrom okolja `frame`.
\begin{frame}{Posebnosti prosojnic}
	    Za prosojnice je značilna uporaba okolja \texttt{frame},
	    s katerim definiramo posamezno prosojnico, \pause
	    postopno odkrivanje prosojnic, \pause
	    ter nekateri drugi ukazi, ki jih najdemo v paketu \texttt{beamer}. \pause
	\begin{exampleblock}{Primer}
		Verjetno ste že opazili, da za naslovno prosojnico niste uporabili
		ukaza \texttt{maketitle}, ampak ukaz \texttt{titlepage}.
	\end{exampleblock}	
\end{frame}


\begin{frame}{Poudarjeni bloki}

	\begin{block}{Opomba}
		Okolja za poudarjene bloke so \texttt{block}, \texttt{exampleblock} in \texttt{alertblock}.
	\end{block}	
	    
    \begin{alertblock}{Pozor!}
		Začetek poudarjenega bloka (ukaz \texttt{begin}) vedno sprejme 
		dva parametra: okolje in naslov bloka.
		Drugi parameter (za naslov) je lahko prazen. 
	\end{alertblock}
		
\end{frame}


\begin{frame}{Tudi v predstavitvah lahko pišemo izreke in dokaze}

	\begin{izrek}
	   Praštevil je neskončno mnogo.
	\end{izrek}

	\begin{proof}
	   Denimo, da je praštevil končno mnogo.
	   	
	   \begin{itemize}[<+->]
		  \item Naj bo $p$ \alert<4>{največje} praštevilo.
		  \item Naj bo $q$ produkt števil $1$, $2$, \ldots, $p$.
		  \item Število $q+1$ ni deljivo z nobenim praštevilom, torej je $q+1$ praštevilo.
		  \item To je protislovje, saj je $q+1>p$. \qedhere
	   \end{itemize}

	\end{proof}

 \end{frame}
 