\documentclass{beamer}

\usepackage{predavanja}
\usepackage{amsfonts}  % Dodajanje paketa za matematične pisave (npr. \mathbb)
\usepackage{tikz}      % Za risanje diagramov znotraj dokumenta
\usetikzlibrary{math}  % Uporaba dodatne knjižnice za matematične operacije v TikZ-u

\usepackage{pgfplots}  % Za risanje grafov znotraj dokumenta
\usepgfplotslibrary{external}  % Omogoči zunanjo obdelavo grafov za hitrejše kompiliranje

\usepackage{array}  % Omogoči napredno delo s tabelami

\theoremstyle{definition}
\newtheorem{definicija}{Definicija}  % Definicija za okolje definicija
\newtheorem{vaja}{Vaja}
\newtheorem{izrek}{Izrek}  % Definicija za okolje izrek

% Definicija okrajšav za cela, realna in kompleksna števila
\newcommand{\ZZ}{\mathbb{Z}}  % Cela števila
\newcommand{\RR}{\mathbb{R}}  % Realna števila
\newcommand{\CC}{\mathbb{C}}  % Kompleksna števila

\begin{document}

\title{Matematični izrazi in uporaba paketa \texttt{beamer}}
\subtitle{\emph{Matematičnih} nalog ni treba reševati!}
\institute{Fakulteta za matematiko in fiziko}
\date{}  % Dodajanje praznega datuma

% Prva stran z naslovom
\frame{\titlepage}

% Kratek pregled vsebine
\begin{frame}
    \frametitle{Kratek pregled}
    \tableofcontents %[pausesections]
\end{frame}

% Sekcija 1 - Paket beamer
\section{Paket \texttt{beamer}}
%  Naslov prosojnice lahko naredimo tudi z dodatnim parametrom okolja `frame`.
\begin{frame}{Posebnosti prosojnic}
	\begin{pause}
	Za prosojnice je značilna uporaba okolja \texttt{frame},
	s katerim definiramo posamezno prosojnico,
	postopno odkrivanje prosojnic,
	ter nekateri drugi ukazi, ki jih najdemo v paketu \texttt{beamer}.
	\begin{exampleblock}{Primer}
		Verjetno ste že opazili, da za naslovno prosojnico niste uporabili
		ukaza \texttt{maketitle}, ampak ukaz \texttt{titlepage}.
	\end{exampleblock}
	\end{pause}	
\end{frame}


\begin{frame}{Poudarjeni bloki}

	\begin{block}{Opomba}
		Okolja za poudarjene bloke so \texttt{block}, \texttt{exampleblock} in \texttt{alertblock}.
	\end{block}	
	    
    \begin{alertblock}{Pozor!}
		Začetek poudarjenega bloka (ukaz \texttt{begin}) vedno sprejme 
		dva parametra: okolje in naslov bloka.
		Drugi parameter (za naslov) je lahko prazen. 
	\end{alertblock}
		
\end{frame}


\begin{frame}{Tudi v predstavitvah lahko pišemo izreke in dokaze}

	\begin{izrek}
	   Praštevil je neskončno mnogo.
	\end{izrek}

	\begin{proof}
	   Denimo, da je praštevil končno mnogo.
	   	
	   \begin{itemize}[<+->]
		  \item Naj bo $p$ \alert<4>{največje} praštevilo.
		  \item Naj bo $q$ produkt števil $1$, $2$, \ldots, $p$.
		  \item Število $q+1$ ni deljivo z nobenim praštevilom, torej je $q+1$ praštevilo.
		  \item To je protislovje, saj je $q+1>p$. \qedhere
	   \end{itemize}

	\end{proof}

 \end{frame}
 

% Sekcija 2 - Paketa amsmath in amsfonts
\section{Paketa \texttt{amsmath} in \texttt{amsfonts}}
\input{prosojnice/2-paketa-amsmath-amsfonts.tex}

% Sekcija 3 - Matematične vsebine (Analiza, logika, množice)
\section[Matematika, 1. del\\\large{Analiza, logika, množice}]{Matematika, 1. del}
\input{prosojnice/3-analiza-logika-mnozice.tex}

% Sekcija 4 - Stolpci in slike
\section{Stolpci in slike}
\input{prosojnice/4-stolpci-slike.tex}

% Sekcija 5 - Paket beamer in tabele
\section{Paket \texttt{beamer} in tabele}
\input{prosojnice/5-beamer-tabele.tex}

% Sekcija 6 - Matematične vsebine (Zaporedja, algebra, grupe)
\section[Matematika, 2. del\\\large{Zaporedja, algebra, grupe}]{Matematika, 2. del}
\input{prosojnice/6-zaporedja-algebra-grupe.tex}

\end{document}