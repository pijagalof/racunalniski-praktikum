\documentclass{beamer}

\usepackage{predavanja}

\theoremstyle{definition}
\newtheorem{definicija}{Definicija}

\theoremstyle{plain}
\newtheorem{izrek}{Izrek}

\title{Matematični izrazi in uporaba paketa \texttt{beamer}}
\subtitle{\emph{Matematičnih} nalog ni treba reševati!}
\institute{Fakulteta za matematiko in fiziko}
\date{}

\begin{document}

\frame{\titlepage}

\begin{frame}
    \frametitle{Kratek pregled}
    \tableofcontents 
\end{frame}


\section{Paket \texttt{beamer}}
%  Naslov prosojnice lahko naredimo tudi z dodatnim parametrom okolja `frame`.
\begin{frame}{Posebnosti prosojnic}
	    Za prosojnice je značilna uporaba okolja \texttt{frame},
	    s katerim definiramo posamezno prosojnico, \pause
	    postopno odkrivanje prosojnic, \pause
	    ter nekateri drugi ukazi, ki jih najdemo v paketu \texttt{beamer}. \pause
	\begin{exampleblock}{Primer}
		Verjetno ste že opazili, da za naslovno prosojnico niste uporabili
		ukaza \texttt{maketitle}, ampak ukaz \texttt{titlepage}.
	\end{exampleblock}	
\end{frame}


\begin{frame}{Poudarjeni bloki}

	\begin{block}{Opomba}
		Okolja za poudarjene bloke so \texttt{block}, \texttt{exampleblock} in \texttt{alertblock}.
	\end{block}	
	    
    \begin{alertblock}{Pozor!}
		Začetek poudarjenega bloka (ukaz \texttt{begin}) vedno sprejme 
		dva parametra: okolje in naslov bloka.
		Drugi parameter (za naslov) je lahko prazen. 
	\end{alertblock}
		
\end{frame}


\begin{frame}{Tudi v predstavitvah lahko pišemo izreke in dokaze}

	\begin{izrek}
	   Praštevil je neskončno mnogo.
	\end{izrek}

	\begin{proof}
	   Denimo, da je praštevil končno mnogo.
	   	
	   \begin{itemize}[<+->]
		  \item Naj bo $p$ \alert<4>{največje} praštevilo.
		  \item Naj bo $q$ produkt števil $1$, $2$, \ldots, $p$.
		  \item Število $q+1$ ni deljivo z nobenim praštevilom, torej je $q+1$ praštevilo.
		  \item To je protislovje, saj je $q+1>p$. \qedhere
	   \end{itemize}

	\end{proof}

 \end{frame}
 

\section{Paketa \texttt{amsmath} in \texttt{amsfonts}}
\begin{frame}{Matrike}
	
	Izračunajte determinanto
	\[
	\begin{vmatrix}
	     -1 & 4 & 4 & -2 \\
		 1 & 4 & 5 & -1 \\
		 1 & 4 & -2 & 2 \\
		 3 & 8 & 4 & 3 \\
	\end{vmatrix}
	\]
	V pomoč naj vam bo Overleaf dokumentacija o matrikah:
	
	\href{https://www.overleaf.com/learn/latex/Matrices}{\beamergotobutton{Matrices}}

\end{frame}


\begin{frame}{Okolje \texttt{align} in \texttt{align*}}

    Dokaži \emph{binomsko formulo}: za vsaki realni števili $a$ in $b$ in za vsako naravno število $n$ velja
		   
	\begin{align*}
        (a+b)^n & = \only<1>{\ldots} \onslide<2,3>{(a+b)(a+b)\dots(a+b)} \\
	\onslide<3>{& = a^n + na^{n-1} b + \dots + \binom{n}{k} a^{n-k} b^k + \dots + n a b^{n-1} + b^n} \\       
	            & = \sum_{k=0}^n \binom{n}{k} a^{n-k} b^k		
	\end{align*}
\end{frame}


\begin{frame}{Še ena uporaba okolja \texttt{align*}}
	
Nariši grafe funkcij:
	\begin{align*}	
	y &= x^2 - 3|x| + 2 &   y &= 3 \sin(\pi+x) - 2 \\
	y &= \log_2(x-2) + 3 &   y &= 2 \sqrt{x^2+15} + 6 \\
	y &= 2^{x-3} + 1    &   y &= \cos(x-3) + \sin^2(x+1) 
    \end{align*}
\end{frame}


\begin{frame}{Okolje \texttt{multline}}
	
	Poišči vse rešitve enačbe
	\begin{multline}
	(1+x+x^2) \cdot (1+x+x^2+x^3+\ldots+x^9+x^{10}) = \\
	= (1+x+x^2+x^3+x^4+x^5+x^6)^2.	
	\end{multline}
\end{frame}


\begin{frame}{Okolje \texttt{cases}}
	
	Dana je funkcija
	\[
	f(x,y) = 
	\begin{cases}
	\frac{3x^2y-y^3}{x^2+y^2};&  (x,y) \neq (0,0), \\
	a; &                         (x,y) = (0,0).
	\end{cases}
	\]

	\begin{itemize} 
	\item Določi $a$, tako da izračunaš limito \( \lim_{(x,y)\to(0,0)} f(x). \)
	\item Izračunaj parcialna odvoda $f_x(x,y)$ in $f_y(x,y)$.
	\end{itemize}
\end{frame}


\section[Matematika, 1. del\\\large{Analiza, logika, množice}]{Matematika, 1. del}
\begin{frame}{Logika in množice}
	\begin{enumerate}
		\item
		Poišči preneksno obliko formule \(
\exists x \colon P(x) \land \forall x \colon Q(x) \Rightarrow \forall x \colon R(x)
\)
.
		\item 
		Definiramo množici \($A$ = \lbrack 2,5 \rbrack\) in \($B$ = \lbrace 0,1,2,3,4\ldots \rbrace\).
		V ravnino nariši:
		\begin{enumerate}
		   \item \(A \cap B \times \emptyset\)
		   \item \((A \cup B)^c \times \mathbb{R}\)
		\end{enumerate}
		\item
		Dokaži:
		\begin{itemize}
			\item \((A \Rightarrow B) \sim (\neg B \Rightarrow \neg A)\)
			\item \(\neg (A \lor B) \sim \neg A \land \neg B \)
		\end{itemize}
	\end{enumerate}
\end{frame}

\begin{frame}{Analiza}
	\begin{enumerate}
		\item
		Pokaži, da je funkcija \(x \mapsto \sqrt{x}\) enakomerno zvezna na \([0,\infty)\).
		\item 
		Katero krivuljo določa sledeč parametričen zapis?
		% Spodaj si pomagajte z dokumentacijo o razmikih v matematičnem načinu.
		% https://www.overleaf.com/learn/latex/Spacing_in_math_mode
		$$
		   x(t) = a \cos t, \quad  
		   y(t) = b \sin t, \quad 
		   t \in [0, 2 \pi]
		$$ 
		\item
		Pokaži, da ima \( f(x) = 3x + sin(2x) \) inverzno funkcijo in izračunaj \((f^-1)'(3\pi)\).
		
		\item
		Izračunaj integral 
		% V rešitvah smo spodnji integral zapisali v vrstičnem načinu,
		% ampak v prikaznem slogu. To naredite tako, da v matematičnem načinu najprej
		% uporabite ukaz displaystyle.
		% Pred dx je presledek: pravi ukaz je \,
		\(\int \frac{2+\sqrt{x+1}}{(x+1)^2-\sqrt{x+1}} dx\)
		%  
		\item 
		Naj bo $g$ zvezna funkcija. Ali posplošeni integral 
		\(\int_{0}^{1} \frac{g(x)}{x^2} dx\)
		konvergira ali divergira? Utemelji.
	\end{enumerate}
\end{frame}

\begin{frame}{Kompleksna števila}
	\begin{enumerate}
		\item
		Naj bo $z$ kompleksno število, $z \ne 1$ in $|z| = 1$.
		Dokaži, da je število \( i \, \frac{z+1}{z-1} \) realno.
		\item
		Poenostavi izraz:
		\[\frac{\frac{3 + i}{2 - 2i} + \frac{7i}{1 - i}}{1 + \frac{i - 1}{4} - \frac{5}{2 - 3i}}\]
	\end{enumerate}
\end{frame}

\section{Stolpci in slike}

\section{Paket \texttt{beamer} in tabele}

\section[Matematika, 2. del\\\large{Zaporedja, algebra, grupe}]{Matematika, 2. del}

\end{document}